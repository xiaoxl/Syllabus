\documentclass[10pt]{amsart}%
\usepackage{lastpage}%
\usepackage{amssymb, amsfonts, latexsym, verbatim, xspace, setspace,tikz,multicol,amsmath,multirow,amsthm,mathrsfs,fancyhdr,bbm,url}%
\textwidth 15.3cm%
\textheight 24cm%
\hoffset=-3.2cm%
\voffset=-1cm%
\oddsidemargin=3.2cm%
\evensidemargin=3.2cm%
\parindent=0pt%
\setlength{\baselineskip}{20pt}%
\renewcommand{\baselinestretch}{1.2}%
%
%
%
\begin{document}%
\normalsize%
\begin{tabular}{|l|l|l|}%
\hline%
\textbf{Date}&\textbf{Contents}&\textbf{Assignments Due Date}\\%
\hline%
Apr. 2&Course intro, \S5.1&HW \S5.1-\S5.3. Due Apr. 13\\%
Apr. 4&\S5.2&\\%
Apr. 5&(D) HW \S5.1-\S5.2&\\%
Apr. 6&\S5.3-\S5.4&\\%
\hline%
Apr. 9&\S5.4-\S5.5&HW \S5.4-\S5.5. Due Apr. 20\\%
Apr. 11&\S5.5&\\%
Apr. 12&(D) HW \S5.3-\S5.5&\\%
Apr. 13&\S5.5&\\%
\hline%
Apr. 16&\S6.1&HW \S6.1-\S6.2. Due Apr. 27\\%
Apr. 18&\S6.1&\\%
Apr. 19&(D) Quiz \#1 (\S5.1-\S5.5)&\\%
Apr. 20&\S6.2&\\%
\hline%
Apr. 23&\S6.2&\\%
Apr. 25&\S6.2&\\%
Apr. 26&(D) HW \S6.1-\S6.2&\\%
Apr. 27&\S6.2-\S7.1&\\%
\hline%
Apr. 30&\S7.1-\S7.2&HW \S7.1-\S7.3 Due May 11\\%
May 2&\S7.2-\S7.3&\\%
May 3&(D) Review for Midterm&\\%
May 4&Midterm&\\%
\hline%
May 7&\S7.3&HW \S7.4-\S7.6. Due May 18\\%
May 9&\S7.4&\\%
May 10&(D) HW \S7.1-\S7.4&\\%
May 11&\S7.4-\S7.5&\\%
\hline%
May 14&\S7.6&HW \S8.1-\S8.2. Due May 25\\%
May 16&\S8.1&\\%
May 17&(D) HW Quiz \#2 (\S7.1-\S7.6)&\\%
May 18&\S8.1&\\%
\hline%
May 21&\S8.2&\\%
May 23&\S8.2&\\%
May 24&(D) HW \S7.5-\S7.6, \S8.1-\S8.2&\\%
May 25&\S8.2&\\%
\hline%
May 28&*Holiday*&HW \S8.3-\S8.4. Due Jun. 8\\%
May 30&\S8.3&\\%
May 31&(D) HW \S8.2-\S8.3&\\%
Jun. 1&\S8.3&\\%
\hline%
Jun. 4&\S8.4&\\%
Jun. 6&\S8.4&\\%
Jun. 7&(D) Review for the final&\\%
Jun. 8&\S8.4&\\%
\hline%
\end{tabular}%


First note that $\sin(n\pi)=0$ for any $n$. So \[a_n=\frac{n\ln n}{n+10}.\]
Then since \[\lim_{n\rightarrow\infty}\frac{n\ln n}{n+10}=\lim_{n\rightarrow\infty}\frac{\frac{n\ln n}{n}}{\frac{n+10}{n}}=\lim_{n\rightarrow\infty}\frac{\ln n}{1+\frac{10}{n}}=\infty,\]
the sequence is divergent.

{\color{red}\bigskip NOTE that since the limit does not exist, by Remark 8.1.12 (page 8 of Note 8.1), we don't know the relation between the limits of the sequence $\displaystyle\frac{n\ln n}{n+10}$ and that of the function $\displaystyle\frac{x\ln x}{x+10}$. Therefore we CANNOT turn to compute the limit of the function $\displaystyle\frac{x\ln x}{x+10}$ and use L'Hospital's Rule to solve it. It is wrong if you try to use L'Hospital's Rule even if it would give the same answer.  }

%
\end{document}